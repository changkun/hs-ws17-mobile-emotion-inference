\section{Methods \& Models}\label{sec:methods}

In this section, we present technical method and models in different data type aspects.

\subsection{Vision Aspect} \label{subsec:vision-model}

The normal RGB camera brings us focusing on how conduct emotion recognition with RGB images. Through depth camera was recently introduced on commercial mobile phone, its principle basically as same as Microsoft Kinect. Considering these two different sensor aspects, we dive into two different research area on vision sensors.


% Landmark detection is the state of the art method and model based on convolutional neural network.
% This is actually related to YOLO algorithm. R-CNN, Fast R-CNN, Faster R-CNN, Mask R-CNN

\cite{Mollahosseini2017}

\subsection{Voice Aspect}\label{subsec:voice-model}

Voice Aspect as we discussed in the previous section, emotion inferring from user speech is basically
processing user speech.

There is another method, which is inferring users' emotions for human-mobile voice dialogue applications.

\subsection{Touch Aspect}\label{subsec:touch-model}

\begin{itemize}
  \item linear model:
  \item feature engineer: \cite{Gao2012} application specific, application context, the recognition rates are very robust even in naturalistic settings in the context of smartphone-based computer games. 
  \item \cite{Shah2015} hand crafted features, for three classes (happy, unhappy, neutral); 
  \item \cite{bhattacharya2017predictive} 7 proposed features, for four classes (Excited, Relaxed, Frustrated, Bored)
  \item \cite{Tikadar2017}: Four discriminative models, namely the Naïve Bayes, K-Nearest Neighbor (KNN), Decision Tree and Support Vector Machine (SVM) were explored, with SVM giving the highest accuracy of 96.75\%.
\end{itemize}

\subsection{Other Aspect}
\label{subsec:other-model}


\begin{itemize}
  \item Vision method: pre-trained ModelNet;
  \item Touch method: artificial feature engineering with support vector machine;
  \item Motion method: artificial feature engineering with support vector machine;
  \item Audio method: Speech to Text with Nature Language Processing, sequence to sequence model;
\end{itemize}


\subsection{Sensors Fusion}\label{subsec:fusion}

% 写作思路:
% 我们首先发现了一篇全面介绍关于技术融合的文章,我们从这篇文章中抽取出移动平台能够使用的部分 
% \cite{Mazzoni2016}

This section should present a table for the combination of previous sensors.
