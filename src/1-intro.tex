\section{Introduction}

% 写作思路:
% 1. 首先介绍情感计算近年来兴起的原因,其次介绍为什么人机交互需要情感计算。
Affective computing is an emerging interdisciplinary research field bringing together researchers and practitioners from various fields, ranging from artificial intelligence, natural language processing, to cognitive and social sciences.

% 2. 然后介绍人类情感的表达方式,以及它们可能通过哪些信息进行采集。\cite{Hertenstein2009} 这篇文章有一副图片非常详细描述了不同的人类情感及其相关表达部位,同时还包含了不同性别之间的差异。

% 3. 接下来在介绍在这个领域的几篇比较完整的综述文章,其中历年来有两篇 \cite{Zhang2014, Politou2017} 有介绍了关于 smartphone sensor 的问题。但是并没有对具体的技术进行相关介绍,同时存在大量的内容丢失;
% 然后一篇 \cite{Garcia-Garcia2017} 详细探讨了普世计算下的 unimodal analysis 到 multimodal analysis,全面分析了不同数据类型之间的结合以及它们所使用的方法,但是并没有考虑传感器ABC,同时它们得出的一些比较好的方法并不一定使用移动设备。

% 4. 于是本文考虑到这样那样的因素,专门讨论在移动设备上多种传感器技术的综合运用,并对其方法、应用和挑战进行介绍。

\cite{Politou2017, Garcia-Garcia2017}

In the following sections, we first introduce how recent researches involve different mobile commodity sensors
to inferring user emotions in Section \ref{sec:mobile}. 
Then we compare 
