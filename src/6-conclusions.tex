\section{Conclusions}

In this paper, we investigated the recent papers among mobile affective computing related to human computer interaction projects.

Section~\ref{sec:source} addresses different data sources in various mobile commodity sensors for emotion inferring in previous studies. These includes: camera, touch screen, motion sensors, microphone, GPS and application context. 

Next, in the Section~\ref{sec:methods}, we first considered the combinations of these types of data. According to our investigation, the following combinations has been applied in mobile affective computing: \textbf{XXX and XXX}; However, \textbf{XXX and XXX are unmined open topics}. Then, we carried out the review of emotion inferring methods based on different type of data source, and compared the tested methods and inferring models from previous researches. In these comparison, we first reviewed various models for user emotion inferring, researchers usually transfers emotion inferring problem into a classification problem or a regression problem. As a classification problem, most researchers consider user emotions can be inferred to three different state (Happy, Unhappy, Neutral, XXXX); Whereas as a regression problem, they define and calculate a continuous variable then try to learn a regression function for emotion inferring based on machine learning method. In each subsection, we highlighted the most useful methods for different type of emotion inferring that concluded by the most recent research papers, they are: XXX and XXX.

In Section~\ref{sec:applications}, we surveys two novel applications in human-computer interaction related projects driven by emotion inferring. XXX considers user emotion as XXX and introduced XXX for XXX. XXXXXXXX

Though we researched the technic scientific approaches of emotion inferring or affective computing in human computer interaction related topics, Section~\ref{sec:challenges} pointed out the current challenges and limitations of this research area. The main challenges of this area are XXXX and XXX. Moreover, the generalisability of affective computing applications are subject to certain limitations. For instance, XXX and XXX.

Nowadays, new technologies and methods provide us new opportunities of affect emotion inferring in an unobtrusive mobile devices. Since the complexity of the interpretation of human behavior at a very deep level is tremendous and requires a highly interdisciplinary collaboration, we believe the true break-throughs application in this field can be established by precisely modeling and new sensing technologies in the future.